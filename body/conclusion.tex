\chapter{总结与展望}

\section{全文总结}

生成式对抗网络(GANs)自提出以来就一直是深度学习乃至人工智能领域的热门话题。在近几年研究人员的持续努力之下,GANs在训练稳定性、生成图像的质量方面取得了长足的进步。随着GANs生成图像真实度越来越接近真实图像,研究人员开始探索GANs的具体应用。本文梳理了GANs的发展历程和应用领域,就图像编辑和多模态图像转换展开了讨论,指出目前所存在的问题,并给出了相应的解决方案。

为了隐空间编辑语义耦合的问题,并更准确地控制生成图像的属性,本文提出了属性一致生成对抗网络(Attribute Consistent Generative Adversarial Networks),简称ACGAN。为了证明ACGAN可基于任意非条件GANs实现,本文在StyleGAN2和文献~\cite{lwgan}提出的轻量级GANs的基础上实现了ACGAN,分别用于人脸属性编辑和自然风景属性编辑。为了全面分析与展示ACGAN在属性编辑方面的优越性,本文对这两种场景下的属性编辑均设计了对应的定量与定性两类实验,其中不乏人脸属性编辑中的属性编辑成功率与保留率,和自然风景属性编辑中属性增益这样原创性的评价指标。为了将ACGAN拓展到现有的二值属性标注数据集,本文还提出了一种属性量化策略,用于生成连续的伪标签。

为了解决难以对多模态输入之间相关性建模的问题,本文提出了一种联合注意力生成式对抗网络 (Joint Attetion Generative Adversarial Networks, JAGAN),探索了如何通过注意力机制通过挖掘模态间的一致性与互补性,提出了模态内注意力和模态间注意力两种注意力机制,高效实现了利用多模态图像补全缺失的模态图像。本文在医学图像转换和面部图像转换两个场景下做了实验,展示了JAGAN相对于现有图像转换方法在多模态场景下的优越性。然后,本文通过消融实验证明了两种注意力机制的有效性,和两种注意力结合使用的必要性。

本文的贡献可归纳为以下几点:

(1)第三章提出了属性一致生成对抗网络ACGAN:GAN的隐空间被分解为内容空间和正交的语义空间,为解决隐空间编辑语义耦合问题提供了理论上的保障。本文提出属性一致性损失,引入属性回归器在训练阶段监督输入属性与生成图像属性之间的一致性。ACGAN统一了GANs生成图像与属性控制两个任务,使生成器天然具有控制生成图像属性的能力。

(2)在第三章中,为了将ACGAN应用于更常见的二值属性数据集,本文设计了一种属性量化方法,来获得连续属性伪标签。这种属性量化方法可以灵活地与现有方法集成以提高属性编辑性能。

(3)第四章提出了联合注意力生成式对抗网络JAGAN。针对多模态图像转换,提出跨模态一致性和模态间互补性的概念,指出现有方法的不足。模态内注意力依据跨模态一致性,借助自表示网络来引导生成器的训练,在特征提取阶段过滤掉无关信息,并模态信息互补提供了所需的特征兼容。模态间注意力依据模态间互补性,为每个输入模态补充生成目标模态所需要的信息。

\section{展望}

虽然我们在第三章提出的 ACGAN 与现有的隐空间编辑和图像转换方法相比实现了更加卓越的性能,但仍存在一些局限性。例如,在计算属性一致性损失时,从均匀分布采样可能会采样到矛盾的属性组合,使用图卷积神经网络(GCN)辅助采样将是非常值得尝试的方案。 此外,对真实图像的编辑十分依赖GAN逆映射(GAN inversion)的效果,这通常需要复杂的 GAN 架构,如StyleGAN2 。然而,使用StyleGAN2 实现ACGAN需要大量的训练时间。第四章提出的 JAGAN 同样存在类似的问题:JAGAN与其他最先进的技术相比实现了卓越的性能,但需要更多的计算资源和计算时间。未来,我们将探索更高效的GAN架构。

除了本文所提到的GANs在图像编辑和多模态图像转换方面的应用外,GANs实际上还有许多方面的应用,如国际计算机视觉大会(ICCV)2019最佳论文SinGAN~\cite{singan}使用单一GANs模型实现了图像相关的5种任务。除图像外,GANs还可用于语音、音乐的合成~\cite{voice1,voice2,voice3}。GANs在特定领域的应用方兴未艾,存在大量问题需要解决,值得所有GANs研究人员继续努力。